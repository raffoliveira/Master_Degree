\documentclass[12pt,a4paper]{article}
\usepackage[utf8]{inputenc}
\usepackage[english]{babel}
\usepackage{amssymb}%usar formula matematica
\usepackage{graphicx}%insere imagem
\usepackage{wrapfig}
\usepackage{float}
\usepackage{hyperref}
\usepackage{indentfirst}
\usepackage{enumitem}
\usepackage{amsmath}
\usepackage{booktabs}
\usepackage{cancel}
\usepackage{subfig}
\usepackage{siunitx}
\usepackage[alf]{abntex2cite} 
%\newtheorem{teo}{Teorema}
\renewcommand{\baselinestretch}{1.5}%estilo da fonte
\usepackage{verbatim} % para comentarios
\usepackage{float}  % para imagens
\usepackage{xcolor}
\usepackage{listings}
\usepackage{geometry}%configura margem
%\geometry{verbose,a4paper,tmargin=30mm,bmargin=20mm,lmargin=30mm,rmargin=20mm}

\begin{document}
\newgeometry{top = 1.5cm, lmargin = 1.5cm, rmargin = 1.5cm} 

\begin{figure}[H]
	\centering
	\begin{minipage}[]{0.07\linewidth}
	\includegraphics[width=\linewidth]{images/ufop.png}	
	\end{minipage}
\hfill
	\begin{minipage}[]{0.6\linewidth}
		\centering
	\textbf{UNIVERSIDADE FEDERAL DE OURO PRETO\\}
		Rafael Francisco de Oliveira - 2021.10171\\
		PCC104 - Projeto e Análise de Algoritmos\\
		Github: \href{https://github.com/raffoliveira/Master_Degree}{raffoliveira}\\
		Lista 2
		
	\end{minipage}
\hfill	
	\begin{minipage}[c]{0.15\linewidth}
	\includegraphics[width=\linewidth]{images/icea.jpg}	
	\end{minipage}

\vspace{0.5cm}
\hrulefill
\end{figure}

{\Large \textbf{Execução das questões práticas}}

\vspace{0.5cm}

A estrutura da pasta denominada \textsf{lista\_2} no GitHub segue abaixo:

A implementação dos algoritmos foi dividida em módulos. A pasta \textsf{exercicios} contém os arquivos de cada exercício. O arquivo chamado \textsf{functions.h} é responsável por ter as declarações de todas as funções enquanto o arquivo chamado \textsf{functions.cpp} contém as implementações de cada função. No momento da execução de cada algoritmo é necessário executar também o arquivo \textsf{functions.cpp}.

A pasta \textsf{testes} contém os arquivos contendo as instâncias para testes. Para a escolha de cada arquivo, basta modificar as linhas \textsf{30} e \textsf{31} do arquivo \textsf{functions.cpp}, as quais especificam o caminho do diretório da pasta e o nome do arquivo escolhido.

O ambiente de execução utilizado durante a criação dos algoritmos foi \textit{Linux}. Por isso, para a execução dos algoritmos, basta executar o seguinte comando abaixo no terminal:

\textbf{g++ exercicio\_1.cpp functions.cpp -o exe}

\textbf{./main}

\vspace{1cm}

{\Large \textbf{Questões teóricas}}

\begin{enumerate}
	\item Apresente um descrição da classe \textit{vector} apresentando o custo computacional de cada uma de suas operações.
	
	\textit{Vector} é uma forma de armazenar informações contíguas, onde os elementos podem ser acessados por interadores (\textit{iterators}) ou por \textit{offsets} de ponteiros para os elementos. A diferença entre um \textit{vector} e um \textit{static array} é que o \textit{vector} é alocado dinamicamente (a memória é alocada a medida que cresce), consequentemente ocupando mais memória enquanto o \textit{static array} alocada memória estática. A tabela abaixo apresenta as suas operações e seus custos operacionais.
	
	\begin{table}[H]
		\centering
		\begin{tabular}{llccll}
			\toprule
			\textbf{Operação} & \textbf{Custo} &&& \textbf{Operação} & \textbf{Custo}\\
			\midrule
			swap & constante &&& clear & linear(\textit{n})\\
			operador= & linear(\textit{n}) &&& assign & linear(\textit{n})\\
			reserve & linear(\textit{n}) &&& shrink\_to\_fit & linear(\textit{n})\\
			erase & linear(\textit{n}) &&& push\_back & constante amortizada\\
			emplace\_back & constante amortizada &&& insert & linear\\
			emplace & linear &&& resize & linear\\
			pop\_back & constant &&& at & constant\\
			get\_allocator & constant &&& operator[] & constant\\
			front & constant &&& back & constant\\
			empty & constant &&& size & constant\\
			max\_size & constant &&& capacity & constant\\
			data & constant\\		
			
			\bottomrule			
		\end{tabular}
	\end{table}
	
	
	
\end{enumerate}

\bibliography{bib/refer.bib}

\end{document}
