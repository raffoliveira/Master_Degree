\documentclass[12pt,a4paper]{article}
\usepackage[utf8]{inputenc}
\usepackage[english]{babel}
\usepackage{amssymb}%usar formula matematica
\usepackage{graphicx}%insere imagem
\usepackage{wrapfig}
\usepackage{float}
\usepackage{hyperref}
\usepackage{indentfirst}
\usepackage{enumitem}
\usepackage{amsmath}
\usepackage{booktabs}
\usepackage{cancel}
\usepackage{subfig}
\usepackage{siunitx}
\usepackage[alf]{abntex2cite} 
%\newtheorem{teo}{Teorema}
\renewcommand{\baselinestretch}{1.5}%estilo da fonte
\usepackage{verbatim} % para comentarios
\usepackage{float}  % para imagens
\usepackage{xcolor}
\usepackage{listings}
\usepackage{geometry}%configura margem
%\geometry{verbose,a4paper,tmargin=30mm,bmargin=20mm,lmargin=30mm,rmargin=20mm}

\begin{document}
\newgeometry{top = 1.5cm, lmargin = 1.5cm, rmargin = 1.5cm} 

\begin{figure}[H]
	\centering
	\begin{minipage}[]{0.07\linewidth}
	\includegraphics[width=\linewidth]{images/ufop.png}	
	\end{minipage}
\hfill
	\begin{minipage}[]{0.6\linewidth}
		\centering
	\textbf{UNIVERSIDADE FEDERAL DE OURO PRETO\\}
		Rafael Francisco de Oliveira - 2021.10171\\
		PCC104 - Projeto e Análise de Algoritmos\\
		Github: \href{https://github.com/raffoliveira/Master_Degree}{raffoliveira}\\
		Lista 2
		
	\end{minipage}
\hfill	
	\begin{minipage}[c]{0.15\linewidth}
	\includegraphics[width=\linewidth]{images/icea.jpg}	
	\end{minipage}

\vspace{0.5cm}
\hrulefill
\end{figure}

{\Large \textbf{Questões práticas}}

\vspace{0.5cm}

A pasta denominada \textsf{lista\_2} no GitHub contém os respectivos exercícios da lista. A implementação dos algoritmos foi dividida em módulos. Abaixo segue a uma breve descrição dos arquivos disponibilizados:

\begin{itemize}
	\item A pasta \textsf{exercicios} contém os arquivos de cada exercício.
	\item O arquivo chamado \textsf{functions.h} é responsável por ter as declarações de todas as funções.
	\item O arquivo chamado \textsf{functions.cpp} contém as implementações de cada função. No momento da execução de cada algoritmo é necessário executar também este arquivo.
	\item A pasta \textsf{testes} contém os arquivos contendo as instâncias para testes. Para a escolha de cada arquivo nesta pasta, basta modificar as linhas \textsf{73} e \textsf{74} do arquivo \textsf{functions.cpp}, as quais especificam o caminho do diretório da pasta e o nome do arquivo escolhido.
\end{itemize}

O ambiente de execução utilizado durante a codificação foi \textit{Linux}. Nenhum tipo de tratamento foi realizado nos dados inseridos pelo usuário. Por isso, siga estritamente as instruções durante a execução. Para a execução dos algoritmos, execute os seguintes comandos abaixo no terminal, mudando apenas o nome do exercício desejado:

\begin{table}[H]
	\centering
	\begin{tabular}{|l|}
		\hline
		\textbf{g++ exercicio\_1.cpp functions.cpp -o exe}\\		
		\textbf{./exe}\\
		\hline
	\end{tabular}
\end{table}

O primeiro comando irá compilar o código e gerar um arquivo executável nomeado de acordo com nome especificado no comando. No exemplo acima, o executável seria renomeado como \textit{exe}. Para executar o executável, basta executar a segunda linha do comando acima especificando o nome do executável criado.

\newpage

{\Large \textbf{Instruções para cada exercício}}

\vspace{0.5cm}

Os exercícios de 1 e 2 irão realizar a leitura dos dados de entrada diretamente dos arquivos da pasta \textsf{testes}. Logo, para as execuções, será apenas necessário escolher o arquivo desejado na linha \textsf{74} do arquivo \textsf{functions.cpp}.

O exercício 3 irá realizar a leitura dos dados de entrada diretamente dos arquivos da pasta \textsf{testes}. Logo, para as execuções, será apenas necessário escolher o arquivo desejado na linha \textsf{74} do arquivo \textsf{functions.cpp}. Porém, necessitrá do dado a ser encontrado no \textit{array}. Esse dado será inserido pelo usuário.

O exercício 4 necessitará da entrada de duas \textit{strings}. A primeira será a \textit{string} original e a segunda será padrão a ser encontrado. Esses dois dados serão inseridos pelo usuário. Para isso, siga as instruções durante a execução.

Os exercícios 5 e 6 necessitarão de um conjunto de pontos cartesianos. O usuário irá inserir todos os pontos. Primeiro irá especificar o número de pontos a serem inseridos, sendo necessário ser maior ou igual a 2. Depois insere-se a coordenada X e depois a coordenada Y. 

O exercício 7 contém quatro matrizes de entradas (3, 4, 5 e 6 cidades). As matrizes correspondem as distâncias entre as cidades. O usuário deve mudar a variável \textsf{NUMBER\_CITY} que se encontra na linha 11 do arquivo \textsf{exercicio\_7.cpp} que corresponde ao número de cidades como entrada para o algoritmo \textsf{TravellingSalesmanProblem()}. Depois remova os comentários das linhas de instruções que realiza a criação da matriz.

Os exercícios 9 e 10 necessitarão da entrada de um grafo. O usuário irá inserir o número de vértices e o número de arestas do grafo. Após esses dados, as conexões serão inseridas. Por exemplo, o vértice 0 está conectado com o vértice 2 e o vértice 2 está conectado com o vértice 3. Depois das conexões, o usuário irá inserir o número a ser encontrado e o vértice inicial de busca. \\
OBS.: o número de vértices influencia os números correspondentes de cada vértice (rótulo), logo se o número de vértices for 5, os vértices podem ser rotulado de 0 a 4. Se o número de vértices for 8, os vértices podem ser rotulado de 0 a 7. Caso contrário, irá causar erro de \textit{segmentation fault}.





\vspace{1cm}

{\Large \textbf{Questões teóricas}}

\begin{enumerate}
	\item Apresente um descrição da classe \textit{vector} apresentando o custo computacional de cada uma de suas operações.
	
	\textit{Vector} é uma forma de armazenar informações contíguas, onde os elementos podem ser acessados por interadores (\textit{iterators}) ou por \textit{offsets} de ponteiros para os elementos. A diferença entre um \textit{vector} e um \textit{static array} é que o \textit{vector} é alocado dinamicamente (a memória é alocada a medida que cresce), consequentemente ocupando mais memória enquanto o \textit{static array} aloca uma memória estática. A tabela abaixo apresenta as suas operações e seus custos operacionais.
	
	\begin{table}[H]
		\centering
		\begin{tabular}{llccll}
			\toprule
			\textbf{Operação} & \textbf{Custo} &&& \textbf{Operação} & \textbf{Custo}\\
			\midrule
			swap & constante &&& clear & linear(\textit{n})\\
			operator= & linear(\textit{n}) &&& assign & linear(\textit{n})\\
			operator[ ] & constant &&& shrink\_to\_fit & linear(\textit{n})\\
			erase & linear(\textit{n}) &&& push\_back & constante amortizada\\
			emplace\_back & constante amortizada &&& insert & (\textit{n})linear\\
			emplace & linear(\textit{n}) &&& resize & linear(\textit{n})\\
			pop\_back & constant &&& at & constant\\
			get\_allocator & constant &&& reserve & linear(\textit{n})\\
			front & constant &&& back & constant\\
			empty & constant &&& size & constant\\
			max\_size & constant &&& capacity & constant\\
			data & constant &&& destructor & linear(\textit{n})\\	
			construct & constant/linear(\textit{n}) &&& begin/cbegin & constant\\
			end/cend & constant &&& rbegin/crbegin & constant\\
			rend/crend & constant &&& swap & constant\\
			erase/erase\_if & linear(\textit{n}) &&& operator== & constant\\
			operator!= & constant &&& operator> & linear(\textit{n})\\
			operator>= & linear(\textit{n}) &&& operator< & linear(\textit{n})\\ 	
			operator<= & linear(\textit{n}) &&& operator<=> & linear(\textit{n})\\ 
			
			\bottomrule			
		\end{tabular}
	\end{table}
	
	
	
\end{enumerate}

\bibliography{bib/refer.bib}

\end{document}
