\documentclass[12pt,a4paper]{article}
\usepackage[utf8]{inputenc}
\usepackage[english]{babel}
\usepackage{amssymb}%usar formula matematica
\usepackage{graphicx}%insere imagem
\usepackage{wrapfig}
\usepackage{float}
\usepackage{hyperref}
\usepackage{indentfirst}
\usepackage{enumitem}
\usepackage{amsmath}
\usepackage{booktabs}
\usepackage{amssymb}
\usepackage{cancel}
\usepackage{subfig}
\usepackage{multirow}
\usepackage{siunitx}
\usepackage[alf]{abntex2cite} 
%\newtheorem{teo}{Teorema}
\renewcommand{\baselinestretch}{1.5}%estilo da fonte
\usepackage{verbatim} % para comentarios
\usepackage{float}  % para imagens
\usepackage{xcolor}
\usepackage{listings}
\usepackage{geometry}%configura margem
%\geometry{verbose,a4paper,tmargin=30mm,bmargin=20mm,lmargin=30mm,rmargin=20mm}

\begin{document}
\newgeometry{top = 1.5cm, lmargin = 1.5cm, rmargin = 1.5cm} 

\begin{figure}[H]
	\centering
	\begin{minipage}[]{0.07\linewidth}
	\includegraphics[width=\linewidth]{images/ufop.png}	
	\end{minipage}
\hfill
	\begin{minipage}[]{0.6\linewidth}
		\centering
	\textbf{UNIVERSIDADE FEDERAL DE OURO PRETO\\}
		Rafael Francisco de Oliveira - 2021.10171\\
		PCC104 - Projeto e Análise de Algoritmos\\
		Github: \href{https://github.com/raffoliveira/Master_Degree}{raffoliveira}\\
		Lista 3
		
	\end{minipage}
\hfill	
	\begin{minipage}[c]{0.15\linewidth}
	\includegraphics[width=\linewidth]{images/icea.jpg}	
	\end{minipage}

\vspace{0.5cm}
\hrulefill
\end{figure}

{\Large \textbf{Questões práticas}}

\vspace{0.5cm}

A pasta denominada \textsf{lista\_3} no GitHub contém os respectivos exercícios da lista. A implementação dos algoritmos foi dividida em módulos. Abaixo segue a uma breve descrição dos arquivos disponibilizados:

\begin{itemize}
	\item A pasta \textsf{exercicios} contém os arquivos de cada exercício.
	\item O arquivo chamado \textsf{functions.h} é responsável por ter as declarações de todas as funções.
	\item O arquivo chamado \textsf{functions.cpp} contém as implementações de cada função. 
	\item A pasta \textsf{testes} contém os arquivos contendo as instâncias para testes. Para a escolha de cada arquivo nesta pasta, basta modificar as linhas \textsf{73} e \textsf{74} do arquivo \textsf{functions.cpp}  para o tipo \textsf{float} e as linhas \textsf{99} e \textsf{100} do arquivo \textsf{functions.cpp}  para o tipo \textsf{string}, as quais especificam o caminho do diretório da pasta e o nome do arquivo escolhido, respectivamente. 
\end{itemize}

O ambiente de execução utilizado durante a codificação foi \textit{Linux}. Nenhum tipo de tratamento foi realizado nos dados inseridos pelo usuário. Por isso, siga estritamente as instruções durante a execução. Para a execução dos algoritmos, execute os seguintes comandos abaixo no terminal, mudando apenas o nome do exercício desejado:

\begin{table}[H]
	\centering
	\begin{tabular}{|l|}
		\hline
		\textbf{g++ exercicio\_1.cpp -o exe}\\		
		\textbf{./exe}\\
		\hline
	\end{tabular}
\end{table}

O primeiro comando irá compilar o código e gerar um arquivo executável nomeado de acordo com nome especificado no comando. No exemplo acima, o executável seria renomeado como \textit{exe}. Para executar o executável, basta executar a segunda linha do comando acima especificando o nome do executável criado.

\newpage

{\Large \textbf{Instruções para cada exercício}}

\vspace{0.5cm}


O exercício \textbf{1} irá realizar a leitura dos dados de entrada diretamente dos arquivos da pasta \textsf{testes}. Logo, para as execuções, será apenas necessário escolher o arquivo desejado de acordo com as instruções anteriores. Nos códigos de cada exercício há linhas comentadas. Remova os comentários e comente as outras linhas semelhantes caso queira modificar o tipo de entrada e a execução.

O exercício \textbf{3} necessitará da entrada de um número inteiro para gerar as permutações. Esse dado será inserido pelo usuário. Para isso, siga as instruções durante a execução.

O exercício \textbf{4} necessitará da entrada de um número inteiro indicando o número de elementos a serem inseridos e posteriormente a entrada de cada elemento. Esses dados serão inseridos pelo usuário. Para isso, siga as instruções durante a execução.





\end{document}
